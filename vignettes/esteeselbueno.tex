% Options for packages loaded elsewhere
\PassOptionsToPackage{unicode}{hyperref}
\PassOptionsToPackage{hyphens}{url}
%
\documentclass[
]{article}
\title{Untitled}
\author{Ines Garcia}
\date{8/1/2022}

\usepackage{amsmath,amssymb}
\usepackage{lmodern}
\usepackage{iftex}
\ifPDFTeX
  \usepackage[T1]{fontenc}
  \usepackage[utf8]{inputenc}
  \usepackage{textcomp} % provide euro and other symbols
\else % if luatex or xetex
  \usepackage{unicode-math}
  \defaultfontfeatures{Scale=MatchLowercase}
  \defaultfontfeatures[\rmfamily]{Ligatures=TeX,Scale=1}
\fi
% Use upquote if available, for straight quotes in verbatim environments
\IfFileExists{upquote.sty}{\usepackage{upquote}}{}
\IfFileExists{microtype.sty}{% use microtype if available
  \usepackage[]{microtype}
  \UseMicrotypeSet[protrusion]{basicmath} % disable protrusion for tt fonts
}{}
\makeatletter
\@ifundefined{KOMAClassName}{% if non-KOMA class
  \IfFileExists{parskip.sty}{%
    \usepackage{parskip}
  }{% else
    \setlength{\parindent}{0pt}
    \setlength{\parskip}{6pt plus 2pt minus 1pt}}
}{% if KOMA class
  \KOMAoptions{parskip=half}}
\makeatother
\usepackage{xcolor}
\IfFileExists{xurl.sty}{\usepackage{xurl}}{} % add URL line breaks if available
\IfFileExists{bookmark.sty}{\usepackage{bookmark}}{\usepackage{hyperref}}
\hypersetup{
  pdftitle={Untitled},
  pdfauthor={Ines Garcia},
  hidelinks,
  pdfcreator={LaTeX via pandoc}}
\urlstyle{same} % disable monospaced font for URLs
\usepackage[margin=1in]{geometry}
\usepackage{color}
\usepackage{fancyvrb}
\newcommand{\VerbBar}{|}
\newcommand{\VERB}{\Verb[commandchars=\\\{\}]}
\DefineVerbatimEnvironment{Highlighting}{Verbatim}{commandchars=\\\{\}}
% Add ',fontsize=\small' for more characters per line
\usepackage{framed}
\definecolor{shadecolor}{RGB}{248,248,248}
\newenvironment{Shaded}{\begin{snugshade}}{\end{snugshade}}
\newcommand{\AlertTok}[1]{\textcolor[rgb]{0.94,0.16,0.16}{#1}}
\newcommand{\AnnotationTok}[1]{\textcolor[rgb]{0.56,0.35,0.01}{\textbf{\textit{#1}}}}
\newcommand{\AttributeTok}[1]{\textcolor[rgb]{0.77,0.63,0.00}{#1}}
\newcommand{\BaseNTok}[1]{\textcolor[rgb]{0.00,0.00,0.81}{#1}}
\newcommand{\BuiltInTok}[1]{#1}
\newcommand{\CharTok}[1]{\textcolor[rgb]{0.31,0.60,0.02}{#1}}
\newcommand{\CommentTok}[1]{\textcolor[rgb]{0.56,0.35,0.01}{\textit{#1}}}
\newcommand{\CommentVarTok}[1]{\textcolor[rgb]{0.56,0.35,0.01}{\textbf{\textit{#1}}}}
\newcommand{\ConstantTok}[1]{\textcolor[rgb]{0.00,0.00,0.00}{#1}}
\newcommand{\ControlFlowTok}[1]{\textcolor[rgb]{0.13,0.29,0.53}{\textbf{#1}}}
\newcommand{\DataTypeTok}[1]{\textcolor[rgb]{0.13,0.29,0.53}{#1}}
\newcommand{\DecValTok}[1]{\textcolor[rgb]{0.00,0.00,0.81}{#1}}
\newcommand{\DocumentationTok}[1]{\textcolor[rgb]{0.56,0.35,0.01}{\textbf{\textit{#1}}}}
\newcommand{\ErrorTok}[1]{\textcolor[rgb]{0.64,0.00,0.00}{\textbf{#1}}}
\newcommand{\ExtensionTok}[1]{#1}
\newcommand{\FloatTok}[1]{\textcolor[rgb]{0.00,0.00,0.81}{#1}}
\newcommand{\FunctionTok}[1]{\textcolor[rgb]{0.00,0.00,0.00}{#1}}
\newcommand{\ImportTok}[1]{#1}
\newcommand{\InformationTok}[1]{\textcolor[rgb]{0.56,0.35,0.01}{\textbf{\textit{#1}}}}
\newcommand{\KeywordTok}[1]{\textcolor[rgb]{0.13,0.29,0.53}{\textbf{#1}}}
\newcommand{\NormalTok}[1]{#1}
\newcommand{\OperatorTok}[1]{\textcolor[rgb]{0.81,0.36,0.00}{\textbf{#1}}}
\newcommand{\OtherTok}[1]{\textcolor[rgb]{0.56,0.35,0.01}{#1}}
\newcommand{\PreprocessorTok}[1]{\textcolor[rgb]{0.56,0.35,0.01}{\textit{#1}}}
\newcommand{\RegionMarkerTok}[1]{#1}
\newcommand{\SpecialCharTok}[1]{\textcolor[rgb]{0.00,0.00,0.00}{#1}}
\newcommand{\SpecialStringTok}[1]{\textcolor[rgb]{0.31,0.60,0.02}{#1}}
\newcommand{\StringTok}[1]{\textcolor[rgb]{0.31,0.60,0.02}{#1}}
\newcommand{\VariableTok}[1]{\textcolor[rgb]{0.00,0.00,0.00}{#1}}
\newcommand{\VerbatimStringTok}[1]{\textcolor[rgb]{0.31,0.60,0.02}{#1}}
\newcommand{\WarningTok}[1]{\textcolor[rgb]{0.56,0.35,0.01}{\textbf{\textit{#1}}}}
\usepackage{graphicx}
\makeatletter
\def\maxwidth{\ifdim\Gin@nat@width>\linewidth\linewidth\else\Gin@nat@width\fi}
\def\maxheight{\ifdim\Gin@nat@height>\textheight\textheight\else\Gin@nat@height\fi}
\makeatother
% Scale images if necessary, so that they will not overflow the page
% margins by default, and it is still possible to overwrite the defaults
% using explicit options in \includegraphics[width, height, ...]{}
\setkeys{Gin}{width=\maxwidth,height=\maxheight,keepaspectratio}
% Set default figure placement to htbp
\makeatletter
\def\fps@figure{htbp}
\makeatother
\setlength{\emergencystretch}{3em} % prevent overfull lines
\providecommand{\tightlist}{%
  \setlength{\itemsep}{0pt}\setlength{\parskip}{0pt}}
\setcounter{secnumdepth}{-\maxdimen} % remove section numbering
\ifLuaTeX
  \usepackage{selnolig}  % disable illegal ligatures
\fi

\begin{document}
\maketitle

\hypertarget{r-markdown}{%
\subsection{R Markdown}\label{r-markdown}}

This is an R Markdown document. Markdown is a simple formatting syntax
for authoring HTML, PDF, and MS Word documents. For more details on
using R Markdown see \url{http://rmarkdown.rstudio.com}.

When you click the \textbf{Knit} button a document will be generated
that includes both content as well as the output of any embedded R code
chunks within the document. You can embed an R code chunk like this:

\begin{Shaded}
\begin{Highlighting}[]
\FunctionTok{summary}\NormalTok{(cars)}
\end{Highlighting}
\end{Shaded}

\begin{verbatim}
##      speed           dist       
##  Min.   : 4.0   Min.   :  2.00  
##  1st Qu.:12.0   1st Qu.: 26.00  
##  Median :15.0   Median : 36.00  
##  Mean   :15.4   Mean   : 42.98  
##  3rd Qu.:19.0   3rd Qu.: 56.00  
##  Max.   :25.0   Max.   :120.00
\end{verbatim}

\hypertarget{including-plots}{%
\subsection{Including Plots}\label{including-plots}}

You can also embed plots, for example:

\includegraphics{esteeselbueno_files/figure-latex/pressure-1.pdf}

Note that the \texttt{echo\ =\ FALSE} parameter was added to the code
chunk to prevent printing of the R code that generated the plot. \#\#
Introduction

\hypertarget{common-cause-analysis}{%
\subsection{Common cause analysis}\label{common-cause-analysis}}

In order to understand and use the examples showed below, it is required
to import.

\begin{Shaded}
\begin{Highlighting}[]
\FunctionTok{library}\NormalTok{(dagitty)}
\FunctionTok{library}\NormalTok{(car)}
\end{Highlighting}
\end{Shaded}

\begin{verbatim}
## Loading required package: carData
\end{verbatim}

\begin{Shaded}
\begin{Highlighting}[]
\FunctionTok{library}\NormalTok{(rethinking)}
\end{Highlighting}
\end{Shaded}

\begin{verbatim}
## Loading required package: rstan
\end{verbatim}

\begin{verbatim}
## Loading required package: StanHeaders
\end{verbatim}

\begin{verbatim}
## Loading required package: ggplot2
\end{verbatim}

\begin{verbatim}
## rstan (Version 2.21.3, GitRev: 2e1f913d3ca3)
\end{verbatim}

\begin{verbatim}
## For execution on a local, multicore CPU with excess RAM we recommend calling
## options(mc.cores = parallel::detectCores()).
## To avoid recompilation of unchanged Stan programs, we recommend calling
## rstan_options(auto_write = TRUE)
\end{verbatim}

\begin{verbatim}
## Do not specify '-march=native' in 'LOCAL_CPPFLAGS' or a Makevars file
\end{verbatim}

\begin{verbatim}
## Loading required package: parallel
\end{verbatim}

\begin{verbatim}
## rethinking (Version 2.13)
\end{verbatim}

\begin{verbatim}
## 
## Attaching package: 'rethinking'
\end{verbatim}

\begin{verbatim}
## The following object is masked from 'package:car':
## 
##     logit
\end{verbatim}

\begin{verbatim}
## The following object is masked from 'package:stats':
## 
##     rstudent
\end{verbatim}

\begin{Shaded}
\begin{Highlighting}[]
\ControlFlowTok{if}\NormalTok{(}\SpecialCharTok{!}\FunctionTok{suppressWarnings}\NormalTok{(}\FunctionTok{require}\NormalTok{(}\StringTok{"rethinking"}\NormalTok{, }\AttributeTok{quietly =} \ConstantTok{TRUE}\NormalTok{))) \{}
\NormalTok{  drawdag }\OtherTok{\textless{}{-}}\NormalTok{ plot}
\NormalTok{\}}
\end{Highlighting}
\end{Shaded}

\#Imagine you want to study what variables influence the Z variable,
having as well \#X and Y as covariates \#You may want to consider the
DAG:

scenario1.DAG \textless- dagitty(``dag \{ X -\textgreater{} Y X
-\textgreater{} Z e\_y -\textgreater{} Y e\_z -\textgreater{} Z \}'')

coordinates(scenario1.DAG) \textless- list(x = c(Y = 1, X = 2, Z = 3,
e\_y = 0.75, e\_z = 2.75), y = c(Y = 3, X = 1, Z = 3, e\_y = 2.75, e\_z
= 2.75))

drawdag(scenario1.DAG)

\#As well as the DAG:

scenario2.DAG \textless- dagitty(``dag \{ X -\textgreater{} Y X
-\textgreater{} Z Y -\textgreater{} Z e\_y -\textgreater{} Y e\_z
-\textgreater{} Z \}'')

coordinates(scenario2.DAG) \textless- list(x = c(Y = 1, X = 2, Z = 3,
e\_y = 0.75, e\_z = 2.75), y = c(Y = 3, X = 1, Z = 3, e\_y = 2.75, e\_z
= 2.75)) drawdag(scenario2.DAG) \#In both cases X is common cause of Y
and Z. In the first case, Y has a causal effect \#on Z, while in the
second case they are independent.

\#How can you know which is the case? How do you know if you want to
consider X, Y or both? \#Let's analyze all the possibilities!

\#Let's create a function that creates the different datasets that we
need. create.dataset \textless- function(b\_yz, N = 500, b\_xy = 3,
b\_xz = 3, e\_x = 1, e\_y = 1, e\_z = 1) \{ name\_df \textless-
data.frame(X = runif(N, 1, 100) + rnorm(N, sd = e\_x))
name\_df\(Y <- name_df\)X * b\_xy + rnorm(N, sd = e\_y)
name\_df\(Z <- name_df\)X * b\_xz + name\_df\$Y * b\_yz + rnorm(N, sd =
e\_z) return(name\_df) \}

\#These functions will be handy later create.datasetv2 \textless-
function(b\_ax, b\_yz=0, N = 500, b\_xy = 3, b\_xz = 3, e\_x = 1, e\_y =
1, e\_z = 1) \{ name\_df \textless- data.frame(A = runif(N, 1, 100) +
rnorm(N)) name\_df\(X <- name_df\)A * b\_ax + rnorm(N, sd = e\_x)
name\_df\(Y <- name_df\)X * b\_xy + rnorm(N, sd = e\_y)
name\_df\(Z <- name_df\)X * b\_xz + name\_df\$Y * b\_yz + rnorm(N, sd =
e\_z) return(name\_df) \}

create.datasetv3 \textless- function(b\_by, b\_bz, b\_yz=(-3), N = 500,
b\_xy = 3, b\_xz = 3, e\_x = 1, e\_y = 1, e\_z = 1, e\_b = 1) \{
name\_df \textless- data.frame(B = runif(N, 1, 100) + rnorm(N, sd =
e\_b)) name\_df\(X <- runif(N, 1, 100) + rnorm(N, sd = e_x)  name_df\)Y
\textless- name\_df\(X * b_xy + name_df\)B * b\_by +rnorm(N, sd = e\_y)
name\_df\(Z <- name_df\)X * b\_xz + name\_df\(Y * b_yz +  name_df\)B *
b\_bz+ rnorm(N, sd = e\_z) return(name\_df) \}

Ynoinfluences \textless- create.dataset(0) Yinfluences \textless-
create.dataset(-2) non.influences \textless- create.dataset(0, b\_xz =
0)

\#We create a function that checks which is the scenario. We assume that
X is in any \#case a common cause of both Y and Z

\#The argument must be a data frame, which column names are Z (the
variable of study) \#Y and X, the common cause.

Y\_check \textless- function (dataset, conflevel = 0.01) \{

model\_with\_Y \textless- lm(Z\textasciitilde X+Y, data = dataset) p.v.X
\textless-(summary(model\_with\_Y)\(coefficients['X','Pr(>|t|)'])  p.v.Y <- (summary(model_with_Y)\)coefficients{[}`Y',
`Pr(\textgreater\textbar t\textbar)'{]})

if ((p.v.X \textless= conflevel)\&(p.v.Y \textgreater{} conflevel))
\{cat(``The variable of analysis is not influenced by Y\n'') cat(`See
plot\n') scenario1.DAG \textless- dagitty(``dag \{ X -\textgreater{} Y X
-\textgreater{} Z e\_y -\textgreater{} Y e\_z -\textgreater{} Z \}'')

\begin{verbatim}
coordinates(scenario1.DAG) <- list(x = c(Y = 1, X = 2, Z = 3, e_y = 0.75, e_z = 2.75),
                                   y = c(Y = 3, X = 1, Z = 3, e_y = 2.75, e_z = 2.75))

drawdag(scenario1.DAG)
return(invisible(1))
}
\end{verbatim}

if ((p.v.X \textless= conflevel)\&(p.v.Y \textless= conflevel))
\{cat(``The variable of analysis is influenced by both X and Y\n'')
cat(`See plot\n') scenario2.DAG \textless- dagitty(``dag \{ X
-\textgreater{} Y X -\textgreater{} Z Y -\textgreater{} Z e\_y
-\textgreater{} Y e\_z -\textgreater{} Z \}'')

\begin{verbatim}
coordinates(scenario2.DAG) <- list(x = c(Y = 1, X = 2, Z = 3, e_y = 0.75, e_z = 2.75),
                                   y = c(Y = 3, X = 1, Z = 3, e_y = 2.75, e_z = 2.75))

drawdag(scenario2.DAG)
return(invisible(2))}
\end{verbatim}

if ((p.v.X \textgreater{} conflevel)\&(p.v.Y \textgreater{} conflevel))
\{cat(``It seems that neither X or Y affect Z\nYou may want to review
your working model\n'') return(invisible(0))\}

if ((p.v.Y \textless= conflevel)\&(p.v.X \textgreater{} conflevel))
\{cat(`It looks like Y is related to Z, but not Z\nYou may want to
revisit the hypothesis 'X = common cause of Y and Z'')
return(invisible(0))\} \}

a \textless- Y\_check(non.influences) b \textless- Y\_check(Yinfluences)
c \textless- Y\_check(Ynoinfluences) \#d \textless-
Y\_check(wrongcommoncause)

\#Another sensitive question that may arise before the analysis is if we
have identified \#the common cause correctly: the objective of the
analysis is to

\#We will now create a toy example to illustrate the problems of a bad
modeling of \#scenario 1. \#We will study the expression of gene INK4a,
key for melanoma development. It will be \#the outcome variable of Z.
\#It is directly affected by UV radiation. UV radiation can come from
sunbathing, which \#increases the appetite for ice cream consumption
(measured in ml of consumed ice cream) \#You collect data of potentially
cancerous tissue from 100 people, from which you know the \#hours they
have spent in the sun the last year, the amount of consumed ice cream
and \#the expression of INK4a.

b\_xy\_i\_uv\_i \textless- 5 b\_xz\_i\_uv\_i \textless- 10
b\_yz\_i\_uv\_i \textless- 0 samplesize \textless- 100

i\_uv\_i.DAG \textless- dagitty(``dag \{ UV.radiation -\textgreater{}
Ice.cream.consumption UV.radiation -\textgreater{} INK4a \}'')
coordinates(i\_uv\_i.DAG) \textless- list(x = c(Ice.cream.consumption =
1, UV.radiation = 2, INK4a = 3), y = c(Ice.cream.consumption = 3,
UV.radiation = 1, INK4a = 3))

drawdag(i\_uv\_i.DAG)

sc1.comm \textless- function(b\_yz, N, b\_xz, b\_xy, reps = 100,
\ldots)\{ onlyY\_pv \textless- rep(NA, reps) both\_pv \textless-
rep(NA,reps)

onlyX\_coefX \textless- rep(NA,reps) both\_coefX \textless- rep(NA,
reps)

for (i in 1:reps) \{

\begin{verbatim}
dataset <- create.dataset(b_yz, N = N, b_xz = b_xz, b_xy = b_xy, ...)

both <- lm(Z~X+Y, data = dataset)
onlyY <- lm(Z~Y, data = dataset)
onlyX <- lm(Z~X, data = dataset)

onlyY_pv[i] <- summary(onlyY)$coefficients["Y", "Pr(>|t|)"]
both_pv[i] <- summary(both)$coefficients["Y", "Pr(>|t|)"]

onlyX_coefX[i] <- summary(onlyX)$coefficients["X", "Estimate"]
both_coefX[i] <- summary(both)$coefficients["X", "Estimate"]

rm(dataset)
\end{verbatim}

\} cat(`\n Change in relevance of Y on Z\n') cat(`\nWhen Z
\textasciitilde{} Y: \nThe p value of Y is', mean(onlyY\_pv),`\n')
cat(`\nWhen Z \textasciitilde Y + X: \nThe p value of Y is',
mean(both\_pv), `\n')

cat(`\n Change in effect of X over Z')

cat(`\nWhen Z \textasciitilde{} X: \nThe estimate for X is',
mean(onlyX\_coefX),`and its s.d. is', sd(onlyX\_coefX),`\n')
cat(`\nWhen Z \textasciitilde Y + X: \nThe estimate for X is',
mean(both\_coefX), `and its s.d. is', sd(both\_coefX),`\n')
cat(`\nBeing input x -\textgreater{} z:', b\_xz) \#This illustrates how,
even if the estimate of the coefficient for X is similar in both cases,
the variance is higher in the presence of Y op \textless- par(mfrow=
c(2,1), mar = rep(3,4)) hist(onlyX\_coefX, main = `Z \textasciitilde{}
X', xlab = `Effect X over Z') abline(v = b\_xz, col = `red')
hist(both\_coefX, main = `Z \textasciitilde{} X + Y', xlab = `Effect X
over Z') abline(v = b\_xz, col = `red') par(op)

list \textless- list(`onlyY\_pv' = onlyY\_pv, `both\_pv' = both\_pv,
`onlyX\_coefX' = onlyX\_coefX, `both\_coefX' = both\_coefX)
invisible(list) \}

sc1.comm(b\_yz = b\_yz\_i\_uv\_i, N = samplesize, b\_xz =
b\_xz\_i\_uv\_i, b\_xy = b\_xy\_i\_uv\_i)

\#The ice cream consumption is not significant relevant to INK4a
expression \#when Uv radiation is present in the model. UV radiation is
a confounder. \#This corresponds to:
impliedConditionalIndependencies(i\_uv\_i.DAG) \#The power of UV
radiation over INK4a doesn't vary much with the presence or absence of Y
in the model. \#But its variance increases when Y is taken into account.
\#Hence, in scenario 1 we should always condition on the common cause X
or we could see \#a fake, but significant, causal relation between Y and
Z.

\#A variation would be to consider that uv raditaion depends on sun
exposure, \#having the following DAG: b\_ax\_i\_uv\_i2 \textless- 2
b\_xy\_i\_uv\_i2 \textless- 5 b\_xz\_i\_uv\_i2 \textless- 10
b\_yz\_i\_uv\_i2 \textless- 0

i\_uv\_i\_sun.DAG \textless- dagitty(``dag \{ Sun -\textgreater{}
UV.radiation UV.radiation -\textgreater{} Ice.cream.consumption
UV.radiation -\textgreater{} INK4a \}'')

coordinates(i\_uv\_i\_sun.DAG) \textless- list(x =
c(Ice.cream.consumption = 1, UV.radiation = 2, Sun = 2, INK4a = 3), y =
c(Ice.cream.consumption = 3, UV.radiation = 2, Sun = 1, INK4a = 3))
drawdag(i\_uv\_i\_sun.DAG)

\#As the significance of ice cream, Y, was covered in the previous
function, \#it will be skipped in this one. We are interested in knowing
if the sun, A, plays a role \#in the value of INK4a, Z, and how
important is it.

sc1.comm.plusancestor \textless- function(b\_yz, N, b\_xz, b\_xy, b\_ax,
reps = 30, e\_x= 1, \ldots)\{ onlyA\_pvA \textless- rep(NA, reps)
bothXA\_pvA \textless- rep(NA,reps) three\_pvA \textless- rep(NA, reps)

onlyA\_coefA \textless- rep(NA,reps) bothXA\_coefA \textless- rep(NA,
reps) three\_coefA \textless- rep(NA, reps)

onlyX\_coefX \textless- rep(NA,reps) bothXA\_coefX \textless- rep(NA,
reps) three\_coefX \textless- rep(NA, reps)

onlyX\_pvX \textless- rep(NA, reps) bothXA\_pvX \textless- rep(NA, reps)
three\_pvX \textless- rep(NA, reps)

\#set.seed(13) \#can be uncommented for reproducibility for (i in
1:reps) \{

\begin{verbatim}
dataset <- create.datasetv2(b_yz= b_yz, N = N, b_xz = b_xz, b_ax = b_ax,
                            b_xy = b_xy, e_x =e_x, ...)
three <- lm(Z~X+A+Y, data = dataset)
bothXA <- lm(Z~X+A, data = dataset)
onlyA <- lm(Z~A, data = dataset)
onlyX <- lm(Z~X, data = dataset)

onlyA_pvA[i] <- summary(onlyA)$coefficients["A", "Pr(>|t|)"]
bothXA_pvA[i] <- summary(bothXA)$coefficients["A", "Pr(>|t|)"]
three_pvA[i] <- summary(three)$coefficients["A", "Pr(>|t|)"]

onlyA_coefA[i] <- summary(onlyA)$coefficients["A", 'Estimate']
bothXA_coefA[i] <- summary(bothXA)$coefficients["A", "Estimate"]
three_coefA[i] <- summary(three)$coefficients["A", "Estimate"]

onlyX_coefX[i] <- summary(onlyX)$coefficients["X", 'Estimate']
bothXA_coefX[i] <- summary(bothXA)$coefficients["X", "Estimate"]
three_coefX[i] <- summary(three)$coefficients["X", "Estimate"]

onlyX_pvX[i] <- summary(onlyX)$coefficients["X", "Pr(>|t|)"]
bothXA_pvX[i] <- summary(bothXA)$coefficients["X", "Pr(>|t|)"]
three_pvX[i] <- summary(three)$coefficients["X", "Pr(>|t|)"]

rm(dataset)
\end{verbatim}

\}

\#\#\#Changes in A \#p valor de A en los modelos, es relevante o no
cat(`\n\_\_\_\_Change in p value of A on Z\n') cat(`\nWhen Z
\textasciitilde{} A: \nThe p value of A is', mean(onlyA\_pvA),`\n')
cat(`\nWhen Z \textasciitilde{} X + A: \nThe p value of A is',
mean(bothXA\_pvA), `\n') cat(`\nWhen Z \textasciitilde{} Y + X + A:
\nThe p value of A is', mean(three\_pvA), `\n') \#estimate de A con y
sin X cat(`\n\_\_\_\_Effect of A over Z\n') cat(`Input A -\textgreater{}
X:', b\_ax,`\nInput X -\textgreater{} Z:', b\_xz,`\nTotal effect A
-\textgreater{} Z', b\_xz * b\_ax, `\n') cat(`\nWhen Z \textasciitilde{}
A: \nCoefficient of A is', mean(onlyA\_coefA),`and its s.d. is',
sd(onlyA\_coefA),`\n') cat(`\nWhen Z \textasciitilde{} X + A:
\nCoefficient of A is', mean(bothXA\_coefA),`and its s.d. is',
sd(bothXA\_coefA),`\n') cat(`\nWhen Z \textasciitilde{} Y + X + A:
\nCoefficient of A is', mean(three\_coefA),`and its s.d. is',
sd(three\_coefA), `\nSee plots:\n') op \textless- par(mfrow= c(2,3), mar
= rep(2,4)) hist(onlyA\_pvA, main = `Z \textasciitilde{} A', xlab = `p
value of A') hist(bothXA\_pvA, main = `Z \textasciitilde{} X + A', xlab
= `p value of A') hist(three\_pvA, main=`Z \textasciitilde X + A + Y',
xlab = `p value of A')

hist(onlyA\_coefA, main = `Z \textasciitilde{} A', xlab = `Effect A over
Z') abline(v = b\_xz\emph{b\_ax, col = `red') hist(bothXA\_coefA, main =
`Z \textasciitilde{} X + A', xlab = `Effect A over Z') abline(v =
b\_xz}b\_ax, col = `red') hist(three\_coefA, main=`Z \textasciitilde X +
A + Y', xlab = `Effect A over Z') abline(v = b\_xz*b\_ax, col = `red')
par(op) \#\#\#Changes in X \#p value \#\#\#\#\#\#it is very obvious
\#cat(`\n\_\_\_\_Change in p value of X on Z\n') \#cat(`\nWhen Z
\textasciitilde{} A: \nThe p value of A is', mean(onlyX\_pvX),`\n')
\#cat(`\nWhen Z \textasciitilde{} X +A: \nThe p value of A is',
mean(bothXA\_pvX), `\n') \#cat(`\nWhen Z \textasciitilde{} Y + X + A:
\nThe p value of A is', mean(three\_pvX), `\n')

\#estimate de X en los modelos y error estandar cat(`\n\_\_\_\_Effect of
X over Z\n\n') cat(`Input X -\textgreater{} Z:', b\_xz,`\n')
cat(`\nWhen Z \textasciitilde{} A: \nCoefficient of X is',
mean(onlyX\_coefX),`and its s.d. is', sd(onlyX\_coefX),`\n')
cat(`\nWhen Z \textasciitilde{} X +A: \nCoefficient of X is',
mean(bothXA\_coefX),`and its s.d. is', sd(bothXA\_coefX),`\n')
cat(`\nWhen Z \textasciitilde{} Y + X + A: \nCoefficient of X is',
mean(three\_coefX),`and its s.d. is', sd(three\_coefX),`\nSee plots:\n')

op \textless- par(mfrow= c(2,3), mar = rep(2,4))

hist(onlyX\_pvX, main = `Z \textasciitilde{} X', xlab = `p value of X')
hist(bothXA\_pvX, main = `Z \textasciitilde{} X + A', xlab = `p value of
X') hist(three\_pvX, main=`Z \textasciitilde X + A + Y', xlab = `p value
of X')

hist(onlyX\_coefX, main = `Z \textasciitilde{} X', xlab = `Effect X over
Z') abline(v = b\_xz, col = `red') hist(bothXA\_coefX, main = `Z
\textasciitilde{} X + A', xlab = `Effect X over Z') abline(v = b\_xz,
col = `red') hist(three\_coefX, main=`Z \textasciitilde X + A + Y', xlab
= `Effect X over Z') abline(v = b\_xz, col = `red')

par(op) \}

sc1.comm.plusancestor(b\_yz = b\_yz\_i\_uv\_i2, N = samplesize,
b\_xz=b\_xz\_i\_uv\_i2, b\_ax = b\_ax\_i\_uv\_i2, b\_xy =
b\_xy\_i\_uv\_i2) \#From this it can be concluded that A is only
significant when X is not in the model. \#The total effect of A is only
appreciated in this model as well. \#From this it can be concluded that
the adjustment of A is required only if we are interested \#on the
effect of A over Z. By conditioning by X, A loses its relevance. This is
supported by: impliedConditionalIndependencies(i\_uv\_i\_sun.DAG) \#The
p value of X increases with the complexity of the model, but in any case
it is significant. \#The estimate of X is around the expected even if
complexity is increased, but its standard error gets higher. \#It the
cause of study is X, conditioning by A is detrimental as the standard
error of its coefficient \#increases, though its p value is never below
any sensible significance level by adjusting by other \#covariates.

\#As seen in the cause of the cause previous work from Ram?n D?az
Uriarte, when the standard \#error of X increases the variance of the
estimate doesn't change as much with the \#presence of A on the model.
We are still working on why this happens. sc1.comm.plusancestor(b\_yz =
b\_yz\_i\_uv\_i2, N = samplesize, b\_xz=b\_xz\_i\_uv\_i2, b\_ax =
b\_ax\_i\_uv\_i2, b\_xy=b\_xy\_i\_uv\_i2, e\_x =10)

\#This function also illustrates a key property of causal inference:
same rules for \#simple models (toy example in Z\_X\_Y\_adjust.R) can be
applied to more complex models \#(as in this case).

\hypertarget{section}{%
\subsubsection{}\label{section}}

\#Let's move on to the scenario 2. We will illustrate with another
example what to expect \#conditioning on the different possibilities. We
have concluded that INK4a over expression \#is caused by UV radiation. A
recent study shows that there is also intervention of \#MATP in the
French population in this process. It seems to follow the following
\#DAG. samplesize \textless- 100 b\_xz\_m\_uv\_i \textless- 4
b\_yz\_m\_uv\_i \textless- (-3) b\_xy\_m\_uv\_i \textless- 2

m\_uv\_i.DAG \textless- dagitty(``dag \{ UV.radiation -\textgreater{}
MATP UV.radiation -\textgreater{} INK4a MATP -\textgreater{} INK4a \}'')

coordinates(m\_uv\_i.DAG) \textless- list(x = c(MATP = 1, UV.radiation =
2, INK4a = 3), y = c(MATP = 3, UV.radiation = 1, INK4a = 3))
drawdag(m\_uv\_i.DAG)

\#In this case, we are not just asking the question how UV radiation (X)
influences \#INK4a (Z), but we may also be interested in how MATP (Y)
affects INK4a.

sc2.comm \textless- function(b\_xz, b\_yz, b\_xy, N, reps = 200, \ldots)
\{ onlyY\_pvY \textless- rep(NA, reps) both\_pvY \textless- rep(NA,
reps) onlyX\_coefX \textless- rep(NA,reps) both\_coefX \textless-
rep(NA, reps) onlyY\_coefY \textless- rep(NA,reps) both\_coefY
\textless- rep(NA,reps)

for (i in 1:reps)\{ dataset \textless- create.dataset(N=N, b\_xz =
b\_xz, b\_yz = b\_yz, b\_xy = b\_xy) both \textless-
lm(Z\textasciitilde X + Y, dataset) onlyY \textless-
lm(Z\textasciitilde Y, dataset) onlyX \textless- lm(Z\textasciitilde X,
dataset)

\begin{verbatim}
onlyY_pvY[i] <- summary(onlyY)$coefficients['Y', 'Pr(>|t|)']
both_pvY[i] <- summary(both)$coefficients['Y', 'Pr(>|t|)']
onlyX_coefX[i] <- summary(onlyX)$coefficients['X', 'Estimate']
both_coefX[i] <- summary(both)$coefficients['X', 'Estimate']
onlyY_coefY[i] <- summary(onlyY)$coefficients['Y', 'Estimate']
both_coefY[i] <- summary(both)$coefficients['Y', 'Estimate']

#rm(dataset)
\end{verbatim}

\}

cat(`p value of Y') cat(`\nWhen Z\textasciitilde Y:', mean(onlyY\_pvY))
cat(`\nWhen Z\textasciitilde Y+X', mean(both\_pvY),`\n\n')

cat('\_\_\_Changes in X\n`) cat('When Z \textasciitilde{}
X:\nX coefficient:', mean(onlyX\_coefX), `s.d:', sd(onlyX\_coefX))
cat(`\nWhen Z \textasciitilde{} X + Y:\nX coefficient:',
mean(both\_coefX), `s.d:', sd(both\_coefX))

cat('\n\n\_\_\_Changes in Y\n`) cat('When Z \textasciitilde{}
Y:\nY coefficient:', mean(onlyY\_coefY), `s.d:', sd(onlyY\_coefY))
cat(`\nWhen Z \textasciitilde{} Y + X:\nY coefficient:',
mean(both\_coefY), `s.d:', sd(both\_coefY)) \#\#legend: blue, direct
effect X, red total effect X, green effect Y

op \textless- par(mfrow= c(2,2), mar = rep(3,4)) hist(onlyX\_coefX, main
= `Z \textasciitilde{} X', xlab = `Effect X over Z') abline(v = b\_xz +
b\_yz*b\_xy, col = `blue')\#total effect, it takes into account both
sources of effect abline(v = b\_xz, col = `red')\#direct effect

hist(both\_coefX, main = `Z \textasciitilde{} X + Y', xlab = `Effect X
over Z') abline(v = b\_xz + b\_yz*b\_xy, col = `blue')\#total effect
abline(v = b\_xz, col = `red')\#direct effect

hist(onlyY\_coefY, main = `Z \textasciitilde{} Y', xlab = `Effect X over
Z') abline(v = b\_yz, col = `green') hist(both\_coefY, main = `Z
\textasciitilde{} Y + X', xlab = `Effect X over Z') abline(v = b\_yz,
col = `green') par(op)

\}

sc2.comm(b\_xz = b\_xz\_m\_uv\_i, b\_yz = b\_yz\_m\_uv\_i, b\_xy =
b\_xy\_m\_uv\_i, N = samplesize)

\#In this case, Y, thus, MATP, has a significant p value in both models,
in presence \#and absence of the common cause X, UV radiation. This
makes sense and was expected.

\#On blue it is shown the total effect of X over Z, as it takes into
account the effect \#of X over Y as well. On red, it is shown the direct
effect of X over Z. On green, the effect of Y over \#Z. \#When the
mediator Y is out of the model, the X estimates the total effect over Z,
\#including Y contribution. Only when Y is included it is possible to
discern what is the \#direct effect of X. \#Depending on the case it
would be more interesting to study the total or the direct \#effect of
X. For this case, we argue that to know the total effect would be better
\#because in the human body MATP expression is unavoidable. \#In any
case, when there are two covariates in the model the variance of the
estimate increases

\#To know the effect of Y over Z, X has to be taken into account. When X
is not \#considered, the estimate for Y is biased; for this reason in
this kind of graphs \#X is called a confounder.

\#To condition on Y or not may give unexpected outcomes when looking at
X over Z. \#In this case, if MATP is not considered, it seems that the
total effect is negative. when\_xz\_4 \textless- create.dataset(b\_xz =
b\_xz\_m\_uv\_i, b\_yz = b\_yz\_m\_uv\_i, b\_xy = b\_xy\_m\_uv\_i, N =
samplesize) scatterplot(Z\textasciitilde X, data = when\_xz\_4, main
=`Original case', regLine=TRUE)

\#If we input a higher X-\textgreater Z value sc2.comm(b\_xz =
b\_xz\_m\_uv\_i\emph{10, b\_yz = b\_yz\_m\_uv\_i, b\_xy =
b\_xy\_m\_uv\_i, N = samplesize) when\_xz\_40 \textless-
create.dataset(b\_xz = b\_xz\_m\_uv\_i}10, b\_yz = b\_yz\_m\_uv\_i,
b\_xy = b\_xy\_m\_uv\_i, N = samplesize) scatterplot(Z\textasciitilde X,
data = when\_xz\_40, main = `If UV radiation effect is stronger',
regLine=TRUE) \#This is because the X contribution has a higher impact
over Z than Y in this second case.

\#The estimate for X in the simpler model isn't negative, it's total
effect is lower \#than the direct effect. \#If the Y -\textgreater{} Z
value wasn't negative: sc2.comm(b\_xz = b\_xz\_m\_uv\_i\emph{10, b\_yz =
b\_yz\_m\_uv\_i}(-1), b\_xy = b\_xy\_m\_uv\_i, N = samplesize)
when\_xz\_40andnegative \textless- create.dataset(b\_xz =
b\_xz\_m\_uv\_i\emph{10, b\_yz = b\_yz\_m\_uv\_i}(-1), b\_xy =
b\_xy\_m\_uv\_i, N = samplesize) \#scatterplot(Z\textasciitilde X, data
= when\_xz\_40andnegative, main = `If MATP enhances INK4a',
regLine=TRUE)

\#Then the effect of X, UV radiation, over Z, INK4a, is increased, as it
should be obvious.

\#The study goes on and we discovers that both the expression of MATP
and INK4a \#is also influenced by another key factor, the cortisol
level. This leaves us \#with the following DAG:

m\_uv\_i\_c.DAG \textless- dagitty(``dag \{ UV.radiation -\textgreater{}
MATP UV.radiation -\textgreater{} INK4a Cortisol -\textgreater{} MATP
Cortisol -\textgreater{} INK4a MATP -\textgreater{} INK4a \}'')

coordinates(m\_uv\_i\_c.DAG) \textless- list(x = c(UV.radiation = 1,
MATP = 2, INK4a = 2, Cortisol = 3), y = c(UV.radiation = 1, MATP = 2,
INK4a = 3, Cortisol = 1)) drawdag(m\_uv\_i\_c.DAG)

samplesize \textless- 100 b\_by\_m\_uv\_i\_c \textless- 3
b\_bz\_m\_uv\_i\_c \textless- 2 b\_xz\_m\_uv\_i\_c \textless- 4
b\_xy\_m\_uv\_i\_c \textless- 2 b\_yz\_m\_uv\_i\_c \textless- (-3)

\#From the previous function we have learnt that the condition on the
mediator \#variable Y, MATP, would allow us to know the direct effect of
UV.radiation. If it is not \#present in the model, what we can see is
the total effect. \#The reasoning behing the UV.radiation-MATP-INK4a set
also apply to the Cortisol-MATP-INK4a set. sc2.comm.extraoverY
\textless- function(b\_by, b\_bz, b\_xz, b\_xy, b\_yz, N, reps = 200,
\ldots) \{ \#que variables quiero ver ahora onlyB\_coefB \textless-
rep(NA,reps) bothBX\_coefB \textless- rep(NA, reps) three\_coefB
\textless- rep(NA,reps)

for (i in 1:reps)\{ dataset \textless- create.datasetv3(N=N, b\_xz =
b\_xz, b\_yz = b\_yz, b\_xy = b\_xy, b\_by = b\_by, b\_bz = b\_bz,
\ldots) onlyB \textless- lm(Z \textasciitilde{} B, dataset) bothBX
\textless- lm(Z \textasciitilde{} X + B, dataset) three \textless- lm(Z
\textasciitilde{} Y + X + B, dataset)

\begin{verbatim}
onlyB_coefB[i] <- summary(onlyB)$coefficients['B', 'Estimate']
bothBX_coefB[i] <- summary(bothBX)$coefficients['B', 'Estimate']
three_coefB[i] <- summary(three)$coefficients['B', 'Estimate']

rm(dataset)
\end{verbatim}

\}

cat('\_\_\_Changes in B\n`) cat('When Z \textasciitilde{}
B:\nB coefficient:', mean(onlyB\_coefB), `s.d:', sd(onlyB\_coefB))
cat(`\nWhen Z \textasciitilde{} X + B:\nB coefficient:',
mean(bothBX\_coefB), `s.d:', sd(bothBX\_coefB)) cat(`\nWhen Z
\textasciitilde{} Y + X + B:\nB coefficient:', mean(three\_coefB),
`s.d:', sd(three\_coefB))

\#\#legend: blue, total efffect X, red direct effect X, green effect Y

op \textless- par(mfrow= c(1,3), mar = rep(3,4)) hist(onlyB\_coefB, main
= `Z \textasciitilde{} B', xlab = `Effect B over Z') abline(v = b\_bz,
col = `red')\#direct effect abline(v = b\_bz + b\_yz*b\_by, col =
`blue')\#total effect of B

hist(bothBX\_coefB, main = `Z \textasciitilde{} B + X', xlab = `Effect B
over Z') abline(v = b\_bz, col = `red')\#direct effect abline(v = b\_bz
+ b\_yz*b\_by, col = `blue')\#total effect of B

hist(three\_coefB, main = `Z \textasciitilde{} B + X + Y', xlab =
`Effect B over Z') abline(v = b\_bz, col = `red')\#direct effect
abline(v = b\_bz + b\_yz*b\_by, col = `blue')\#total effect of B par(op)
\}

sc2.comm.extraoverY(b\_by = b\_by\_m\_uv\_i\_c, b\_bz =
b\_bz\_m\_uv\_i\_c, b\_xz = b\_xz\_m\_uv\_i\_c, b\_xy =
b\_xy\_m\_uv\_i\_c, b\_yz = b\_yz\_m\_uv\_i\_c, N = samplesize) \#The
total effect of cortisol, B, is well reflected when it is on its own in
the model or \#when UV radition, X, is considered, because they are
independent from each other \#as can be seen in
impliedConditionalIndependencies(m\_uv\_i\_c.DAG) \#The variance of the
coefficient when it is found along X is smaller than when B (cortisol)
\#is checked on its own or when it also considers the collider Y (MATP).
\#Therefore in this type of graph it would be prefered to consider both
B and X on the model: \#the variance of the estimates is smaller and the
toal effect is calculated. \#As in the previous case, condiotioning on Y
may be counterproductive. \#The absence of unmeasured confounding for
the influence of both the exposure \#and the intermediate variable on
the outcome is required for estimating direct \#effects. If both of
these requirements are not satisfied, no approach can offer \#unbiased
estimates of exposure's direct effects.

\end{document}
